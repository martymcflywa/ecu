\section{Conclusion}

This literature review has explored the differences between genders in the use of SNS, and has identified key differences in user motivation. Studies have shown that men are more likely to perform information-based activities on SNS than women, which align with the social gender role framework introduced by \citet{Eagly1987}. Men are also more open to expanding their networks and use SNS as a tool to create new relationships, and more frequently use SNS as a dating platform than women. Women on the other hand, have been found to use SNS to maintain current relationships, and are generally more predisposed than men to provide emotional support to their friends. Women are attracted to the socialising aspect of SNS, and recent studies have shown that women have a larger social network than men. These key differences are also consistent with social gender theory.

All but one study within the scope of this review used self-reported data from their respondents. \citetapos{Haferkamp2012} research included observed data, and although the results were consistent with findings from other studies, there is insufficient literature within this review to conclude that self-reported data and observed data will yield similar results. There are opportunities to conduct further research based on observed data, however questions regarding user privacy would have to be raised, with the amount of personal information available on SNS.

CMC competency as a variable was not found in any of the studies relating to gender in this review. \citet{Ross2009} measured CMC competency, however the research focused on the effect of personality on SNS users and did not exhibit any correlations between gender.

\citet[p. 897]{Kimbrough2013} succinctly pointed out that while users have the ability to choose to behave in any way they wish online, men and women still conform to behaviour that is consistent with ``social role expectations'' from the offline world. This literature review has identified some motivational differences in the use of SNS and mapped them to gender role expectations with the aim of better understanding the differences between genders in SNS use.