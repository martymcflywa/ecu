\section{Literature Review}

\subsection{Motivational differences between genders}

\citet{Fallows2005} compared American male and female Internet users and found that men were more likely to use the Internet in general for information gathering purposes, while women were more likely to use the Internet for social applications to maintain current relationships. According to the literature in this review, this statement also translates to the use of SNS such as Facebook.

\subsubsection{Gathering information and maintaining relationships}

Widely cited research by \citet{Raacke2008} was among the first to examine the impact of SNS on college students and observed that men, compared to women, were more likely to use SNS to find out about events, indicating that men spend more time than women performing information-gathering activities on SNS. In support of this view, \citet[p. 2]{Choi2014} state that men have higher positive attitudes towards SNS advertising than women, as men perceive such advertising as ``useful information'' due to their ``information-oriented motivation''.

In contrast, \citet{Mazman2011} and \citet{Muscanell2012} both assert that women were more likely to seek out old friends on the network, and are more likely to utilise SNS communication tools to maintain existing relationships. To support this statement, \citet{Joiner2014} provides evidence that women are more likely to demonstrate higher emotional support in response to a friend's negative Facebook status update. Women are also twice as likely to respond publicly to a negative status update when compared to men \citep[p. 167]{Joiner2014}. These findings highlight a difference in motivations of SNS use between genders. 

It is worth noting however, that studies by \citet{Raacke2008}, \citet{Muscanell2012} and \citet{Joiner2014} were limited to participants from a single US college comprised of first-year undergraduate students who provided self-reported estimates. Interestingly, \citet{Raacke2008} and \citet{Joiner2014} gathered data from respondents via paper questionnaires, as opposed to \citetapos{Muscanell2012} online questionnaire method. Online questionnaires have the potential to skew results towards users who may spend more time online, possibly use SNS more, and have higher competency in SNS use, compared to those who spend less time online \citep[p. 280]{Hargittai2007}.

Competency in SNS use, otherwise classed as ``Computer Mediated Communication (CMC) Competency'' by \citet[p. 579]{Ross2009}, was a variable largely ignored in most of the research within the scope of this review, which could ``influence how much people use social networking sites'' \citep[p. 898]{Kimbrough2013}. Without the measurement of CMC competency, it is only assumed that all participants of such studies are equally skilled in the use and application of SNS, which certainly may not be the case, as \citet{Ross2009} suggests.

\citetapos{Choi2014} research provides a unique perspective in the role of gender in Facebook use, comparing the relationship of self-presentation on brand-related word-of-mouth and gender's moderating effects. Again, the research was limited to participant self-reported estimates. The study was also limited to respondents within the ages of ninteeen to thirty-nine, as \citet[p. 3]{Choi2014} claims that age bracket represented the ``primary Facebook user population''.

\subsubsection{Making new relationships}

In contrast to women using SNS as a medium to maintain existing relationships, men have been found to use SNS as a tool for creating new relationships and expanding their networks \citep{Mazman2011}. This view is supported by findings in research by \citet{Muscanell2012}, \citet{Raacke2008} and \citet{Haferkamp2012}, which indicate that men are more likely to use SNS for dating purposes than women, and demonstrates a difference in motivation between genders.

\citetapos{Haferkamp2012} research was based on randomly selected users from \href{http://www.studivz.net}{StudiVZ}, a German SNS for students. At the time of research in 2010, the majority of StudiVZ profiles were public and ``used without privacy settings'' \citep[p. 92]{Haferkamp2012}, which allowed the study of observed data from participant profiles, together with self-reported online questionnaire results. Although \citetapos{Haferkamp2012} research utilized observed data from respondents from another country while using a completely unique SNS, the results were consistent with most of the literature in this review. However, access to public user profiles allowed \citet{Haferkamp2012} to analyse gender differences in profile photo preference which will be discussed again later.

\subsubsection{Profile photos}

Now's a good time to discuss profile photo differences or self-presentation. May need to change up the last section of the intro, remove the three pillars and replace with something else to map what's being covered in the review. 