\subsection{Paging}

\subsubsection*{Babaoglu, Ö., \& Joy, W. (1981). Converting a swap-based system to do paging in an architecture lacking page-referenced bits. In SOSP ’81 Proceedings of the eighth ACM symposium on Operating systems principles (Vol. 15, pp. 78–86). New York. doi:10.1145/800216.806595}

\citet{Babaoglu1981} describe the challenges and their methods to converting UNIX's from a segmented, swap-based system to a paging system, in a computer with architecture that does not support page-referenced bits. The authors explain the use of page-referenced bits as essential to page replacement algorithms such as clock page replacement and sampled working set (SWS). They set out to implement their variation of clock page replacement in UNIX by simulating page-referenced bits through software, after comparing various page replacement algorithms.

Babaoglu and Joy justify their design and optimization decisions by comparing the performance of the clock page replacement, and identify opportunities for improvement, based on the given hardware and operating system. For instance, \citet[p. 80]{Babaoglu1981} state the original algorithm only seeks to replace a single page when triggered by a page fault, and through testing, found examples where page requests spiked due to UNIX's non-uniform operations. They modified the algorithm by implementing a free page pool containing page frames not currently in the clock loop, and set a minimum free page pool size as a threshold. When this threshold is reached, clock page replacement is triggered and pages are replaced until the free page pool size reaches the threshold again. As the free page pool size decreases, the scan rate of the clock page replacement implementation increases until it reaches a maximum scan rate, which is ``determined by the time it takes to simulate the setting of a referenced bit'' \citep[p. 80]{Babaoglu1981}.

The authors compared their clock paging system to a swap-based system and present their findings with graphs comparing performance between the two. They found that under lower load levels, their clock page implementation out performed the swap-based system. However, under heavy load, while clock page replacement exhibited much lower page traffic, the overhead required to support it was higher than the swap-based system. The authors counter this finding by stating that the CPU utilization was greater for the clock paging system compared to the swap-based system.

Babaoglu and Joy conclude the article with a recommended requirement when designing a page replacement algorithm for architecture that does not support page-referenced bits, the results of their comparison of the two memory management algorithms, and limitations of their global clock replacement algorithm.
